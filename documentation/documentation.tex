\documentclass[article]{jss}
\usepackage[utf8]{inputenc}

\providecommand{\tightlist}{%
  \setlength{\itemsep}{0pt}\setlength{\parskip}{0pt}}

\author{
Anna Michalek\\European Central Bank \And Alain Quartier-La-Tente\\Insee
}
\title{\pkg{RJDemetra}: A R Interface To JDemetra+ Seasonal Adjustment Software}

\Plainauthor{Anna Michalek, Alain Quartier-La-Tente}
\Plaintitle{A Capitalized Title: Something about a Package foo}
\Shorttitle{\pkg{RJDemetra}: A Capitalized Title}

\Abstract{
The abstract of the article.
}

\Keywords{\proglang{R}, seasonal adjustment, time series}
\Plainkeywords{R, seasonal adjustment, time series}

%% publication information
%% \Volume{50}
%% \Issue{9}
%% \Month{June}
%% \Year{2012}
%% \Submitdate{}
%% \Acceptdate{2012-06-04}

\Address{
    }

% Pandoc header

\usepackage{amsmath} \usepackage{booktabs} \usepackage{longtable} \usepackage{array} \usepackage{multirow} \usepackage{wrapfig} \usepackage{float} \usepackage{pdflscape} \usepackage{tabu} \usepackage{threeparttable} \usepackage{threeparttablex} \usepackage[normalem]{ulem} \usepackage{makecell}

\begin{document}

\hypertarget{introduction}{%
\section{Introduction}\label{introduction}}

The package \pkg{RJDemetra} provides a R interface to the seasonal
adjustment software JDemetra+. Note that, JDemetra+ being implemented in
Java, \pkg{RJDemetra} relies on the \pkg{rJava} package and Java SE 8 or
later version is required. The two leading seasonal adjustment methods
TRAMO/SEATS+ and X-12ARIMA/X-13ARIMA-SEATS can be used with all the
specifications defined in JDemetra+.

This article is structured as following.In the first section the .. is
presented.

\hypertarget{seasonal-adjustment-in-brief}{%
\subsection{Seasonal adjustment in
brief}\label{seasonal-adjustment-in-brief}}

The \textbf{first step} of seasonal adjustment, both in
X-12ARIMA/X-13ARIMA-SEATS and TRAMO-SEATS+, consists of pre-adjusting
the time series by removing from it the deterministic effects and
estimating missing observations. Among deterministic effects, we
distinguish outliers, calendar and regression effects. In this step,
also forecasts and backcasts of the pre-adjusted series are estimated
which allows applying linear filters at both ends of the series in the
second step of the seasonal adjustment. The pre-adjustment,
linearization, of the input series is achieved with a \textbf{RegARIMA}
model (model with ARIMA errors) as specified below.

\[z_t=y_t\beta+x_t\] where

\begin{itemize}
\tightlist
\item
  \(z_t\) - is the original series;
\item
  \(\beta = (\beta_1,...,\beta_n)\) - a vector of regression
  coefficients;
\item
  \(y_t = (y_{1t},...,y_{nt})\) - \(n\) regression variables (outliers,
  calendar effects, user-defined variables);
\item
  \(x_t\) - a disturbance that follows the general ARIMA process:
\item
  \(\phi(B)\delta(B)x_t=\theta(B)a_t\); \(\phi(B), \delta(B)\) and
  \(\theta(B)\) are the finite polynomials in \(B\); \(a_t\) is a
  white-noise variable with zero mean and a constant variance.
\end{itemize}

The polynomial \(\phi(B)\) is a stationary autoregressive (AR)
polynomial in \(B\), which is a product of the stationary regular AR
polynomial in \(B\) and the stationary seasonal polynomial in \(B^s\):

\[\phi(B)=\phi_p(B)\Phi_{bp}(B^s)=(1+\phi_1B+...+\phi_pB^p)(1+\Phi_1B^s+...+\Phi_{bp}B^{bps}\]

where:

\begin{itemize}
\tightlist
\item
  \(p\) - number of regular AR terms (in the package and in JDemetra+
  \(p \le 3\));
\item
  \(bp\) - number of seasonal AR terms (in the package and in JDemetra+
  \(bp \le 1\));
\item
  \(s\) - number of observations per year (frequency of the time
  series).
\end{itemize}

The polynomial \(\theta(B)\) is an invertible moving average (MA)
polynomial in \(B\), which is a product of the invertible regular MA
polynomial in \(B\) and the invertible seasonal MA polynomial in
\(B^s\):

\[\theta(B)=\theta_q(B)\Theta_{bq}(B^s)=(1+\theta_1B+...+\theta_qB^q)(1+\Theta_1B^s+...+\Theta_{bq}B^{bqs})\]

where:

\begin{itemize}
\tightlist
\item
  \(q\) - number of regular MA terms (in the package and in JDemetra+
  \(q \le 3\));
\item
  \(bq\) - number of seasonal MA terms (in the package and in JDemetra+
  \(bq \le 1\));
\end{itemize}

The polynomial \(\delta(B)\) is the non-stationary AR polynomial in
\(B\) (unit roots):

\[\delta(B)=(1-B)^d(1-B^s)^{d_s}\]

where:

\begin{itemize}
\tightlist
\item
  \(d\) - regular differencing order (in the package and in JDemetra+
  \(d \le 1\));
\item
  \(d_s\) - seasonal differencing order (in the package and in JDemetra+
  \(d_s \le 1\));
\end{itemize}

In the \textbf{second part} of seasonal adjustment, called the
\textbf{decomposition}, the pre-adjusted series is decomposed into the
following components: trend-cycle (\code{t}), seasonal component
(\code{s}) and irregular component (\code{i}). The decomposition can be:

\begin{itemize}
\tightlist
\item
  additive (\code{y = t + s + i})
\item
  multiplicative (\code{y = t * s * i})
\item
  log-additive (\code{log(y) = log(t)+log(s)+log(i)}) or
\item
  pseudo-additive (\code{y = t*(s+i-1)})
\end{itemize}

The last two decompositions are available only under X13.

The method of decomposing the pre-adjusted series differs between
TRAMO-SEATS+ and X-12ARIMA/X-13ARIMA. In TRAMO-SEATS+, SEATS (``Signal
Extraction in ARIMA Time Series'') decomposes the observed series with a
ARIMA-model based method. Whereas in X-12ARIMA/X-13ARIMA, the X11
algorithm decomposes the time series by means of linear filters. More
information on the TRAMO-SEATS+ method can be found on the Bank of Spain
website (\href{www.bde.es}{link}) and on X-12ARIMA/X-13ARIMA, on the
U.S. Census Bureau website.

As a result of seasonal adjustment, the final seasonally adjusted series
(\code{sa}) shall be free of seasonal and calendar-related movements.

More details on the methodlogy used in JDemetra+ can be found in the
JDemetra manuals and user guides available at
\href{https://ec.europa.eu/eurostat/cros/content/documentation_en}{link}.

\hypertarget{rjdemetra-basics}{%
\section{RJDemetra basics}\label{rjdemetra-basics}}

The \pkg{RJDemetra} package alows to:

\begin{itemize}
\tightlist
\item
  create and modify model specifications
\item
  create and modify models
\item
  import/export JDemetra+ workspaces
\end{itemize}

\hypertarget{dataset}{%
\subsection{Dataset}\label{dataset}}

In this package we include the sts\_inpr\_m database of Eurostat, which
contains the monthly industrial production indices in manufacturing in
the European Union. It contains 37 time series from january 1990 to
december 2017 which are considered to be affect by seasonal and working
day effects. The data is a \code{ts} object and can be accessed using
the \code{ipi_c_eu} object. The following snippet of code plot the
industrial production index of the Euro aera:

\begin{CodeChunk}

\begin{CodeInput}
R> library(RJDemetra)
R> plot(ipi_c_eu[, "EA19"])
\end{CodeInput}


\begin{center}\includegraphics{documentation_files/figure-latex/unnamed-chunk-2-1} \end{center}

\end{CodeChunk}

\hypertarget{estimate-a-predefined-regarima-and-sa-model}{%
\section{Estimate a predefined regarima and SA
model}\label{estimate-a-predefined-regarima-and-sa-model}}

As in JDemetra+, \pkg{RJDemetra} package allows to perform seasonal
adjustment using pre-defined model specifications. The specifications
are separately defined for TRAMO-SEATS and X-13ARIMA-SEATS estimation
methods. Also, only the first step of seasonal adjustment - RegARIMA
estimation -- is possible with pre-defined specifications. The
pre-defined model specifications are described in tables
\ref{tab:pre_def_ts} and \ref{tab:pre_def_x13}. They correspond to the
most commonly used specifications and users are recommended to start
their analysis with one of those specification. Pre-defined
specifications are identical for pre-adjustment (column 1) and for
seasonal adjustment (column 2).

\begin{table}

\caption{\label{tab:unnamed-chunk-3}\label{tab:pre_def_ts}Pre-defined specification for TRAMO and TRAMO-SEATS}
\centering
\fontsize{7}{9}\selectfont
\begin{tabular}[t]{c>{\centering\arraybackslash}p{1.cm}>{\centering\arraybackslash}p{1.cm}>{\centering\arraybackslash}p{1.5cm}>{\centering\arraybackslash}p{0.9cm}>{\centering\arraybackslash}p{0.9cm}>{\centering\arraybackslash}p{1.5cm}>{\centering\arraybackslash}p{0.9cm}c}
\toprule
\multicolumn{2}{c}{Specification} & \multicolumn{1}{c}{} \\
\cmidrule(l{2pt}r{2pt}){1-2}
TRAMO & TRAMO-SEATS & Trans-formation & Pre-adjust-ment for leap-year & Working days & Trading days & Easter effect & Outliers & ARIMA model\\
\midrule
TR0 & RSA0 & no & no & no & no & no & no & (0,1,1)(0,1,1)\\
TR1 & RSA1 & test & no & no & no & no & test & (0,1,1)(0,1,1)\\
TR2 & RSA2 & test & no & test & no & test & test & (0,1,1)(0,1,1)\\
TR3 & RSA3 & test & no & no & no & no & test & AMI\\
TR4 & RSA4 & test & no & test & no & test & test & AMI\\
\addlinespace
TR5 & RSA5 & test & no & no & yes & test (Standard) & test & AMI\\
TRfull (default) & RSAfull (default) & test & yes & no & test & test (Include Easter) & test & AMI\\
\bottomrule
\end{tabular}
\end{table}

\begin{table}

\caption{\label{tab:unnamed-chunk-3}\label{tab:pre_def_x13}Pre-defined specification for RegARIMA and X-13ARIMA-SEATS}
\centering
\fontsize{7}{9}\selectfont
\begin{tabular}[t]{c>{\centering\arraybackslash}p{1.7cm}>{\centering\arraybackslash}p{1.cm}>{\centering\arraybackslash}p{1.4cm}>{\centering\arraybackslash}p{0.9cm}>{\centering\arraybackslash}p{0.9cm}>{\centering\arraybackslash}p{0.9cm}>{\centering\arraybackslash}p{0.9cm}c}
\toprule
\multicolumn{2}{c}{Specification} & \multicolumn{1}{c}{} \\
\cmidrule(l{2pt}r{2pt}){1-2}
RegARIMA & X-13ARIMA-SEATS & Trans-formation & Pre-adjust-ment for leap-year & Working days & Trading days & Easter effect & Outliers & ARIMA model\\
\midrule
RG0 & X11 & no & no & no & no & no & no & (0,1,1)(0,1,1)\\
RG1 & RSA1 & test & no & no & no & no & test & (0,1,1)(0,1,1)\\
RG2c & RSA2c & test & test & test & no & test & test & (0,1,1)(0,1,1)\\
RG3 & RSA3 & test & no & no & no & no & test & AMI\\
RG4c & RSA4c & test & test & test & no & test & test & AMI\\
RG5c (default) & RSA5 (default) & test & test & no & test & test & test & AMI\\
\bottomrule
\end{tabular}
\end{table}

The model specification can be defined from an existing model
specification or an estimated model, as each of the estimated model
contains also its specification.

\hypertarget{sa-object-structure}{%
\section{SA object structure}\label{sa-object-structure}}

\begingroup\fontsize{7}{9}\selectfont

\begin{longtable}[t]{lllll}
\caption{\label{tab:unnamed-chunk-4}SA object structure}\\
\toprule
\multicolumn{1}{c}{ } & \multicolumn{1}{c}{ } & \multicolumn{1}{c}{ } & \multicolumn{2}{c}{When adjusted with:} \\
\cmidrule(l{2pt}r{2pt}){4-5}
\multicolumn{1}{c}{\em  } & \multicolumn{1}{c}{\em  } & \multicolumn{1}{c}{\em  } & \multicolumn{1}{c}{\em x13/x13\_def} & \multicolumn{1}{c}{\em  tramoseats/tramoseats\_def} \\
\cmidrule(l{2pt}r{2pt}){4-4} \cmidrule(l{2pt}r{2pt}){5-5}
Object & Level & Type & Class & Class\\
\midrule
sa\_object & 0 & list & SA, X13 & SA, TRAMO\_SEATS\\
\textbf{\hspace{1em}regarima} & \textbf{1} & \textbf{list} & \textbf{regarima, X13} & \textbf{regarima, TRAMO\_SEATS}\\
\hspace{2em}specification & 2 & list &  & \\
\hspace{3em}estimate & 3 & data.frame &  & \\
\hspace{3em}transform & 3 & data.frame &  & \\
\addlinespace
\hspace{3em}regression & 3 & list &  & \\
\hspace{4em}userdef & 4 & list &  & \\
\hspace{5em}specification & 5 & data.frame &  & \\
\hspace{5em}outliers & 5 & data.frame or NA(empty) &  & \\
\hspace{5em}variables & 5 & list &  & \\
\addlinespace
\hspace{6em}series & 6 & mts, ts, matrix or NA(empty) &  & \\
\hspace{6em}description & 6 & data.frame or NA(empty) &  & \\
\hspace{4em}trading.days & 4 & data.frame &  & \\
\hspace{4em}easter & 4 & data.frame &  & \\
\hspace{3em}outliers & 3 & data.frame &  & \\
\addlinespace
\hspace{3em}arima & 3 & list &  & \\
\hspace{4em}specification & 4 & data.frame &  & \\
\hspace{4em}coefficients & 4 & data.frame or NA(empty) &  & \\
\hspace{3em}forecast & 3 & data.frame &  & \\
\hspace{3em}span & 3 & data.frame &  & \\
\addlinespace
\hspace{2em}arma & 2 & vector - numeric &  & \\
\hspace{2em}arima.coefficients & 2 & matrix &  & \\
\hspace{2em}regression.coefficients & 2 & matrix &  & \\
\hspace{2em}loglik & 2 & matrix &  & \\
\hspace{2em}model & 2 & list &  \vphantom{1} & \\
\addlinespace
\hspace{3em}spec\_rslt & 3 & data.frame &  & \\
\hspace{3em}effects & 3 & mts, ts, matrix &  & \\
\hspace{2em}residuals & 2 & ts &  & \\
\hspace{2em}residuals.stat & 2 & list &  & \\
\hspace{3em}st.error & 3 & numeric &  & \\
\addlinespace
\hspace{3em}tests & 3 & data.frame & regarima\_rtests, data.frame & \\
\hspace{2em}forecast & 2 & mts, ts, matrix &  & \\
\textbf{\hspace{1em}decomposition} & \textbf{1} & \textbf{list} & \textbf{decomposition\_X11}\\
\hspace{2em}specification & 2 & data.frame & X11\_spec, data.frame & \\
\hspace{2em}mode & 2 & character &  \vphantom{1} & \\
\addlinespace
\hspace{2em}mstats & 2 & matrix &  & \\
\hspace{2em}si\_ratio & 2 & mts, ts, matrix &  & \\
\hspace{2em}s\_filter & 2 & vector - character &  & \\
\hspace{2em}t\_filter & 2 & character &  & \\
\textbf{\hspace{1em}decomposition} & \textbf{1} & \textbf{list} & \textbf{} & \textbf{decomposition\_SEATS}\\
\addlinespace
\hspace{2em}specification & 2 & data.frame & seats\_spec, data.frame & \\
\hspace{2em}mode & 2 & character &  & \\
\hspace{2em}model & 2 & list &  & \\
\hspace{3em}model & 3 & matrix or empty list &  & \\
\hspace{3em}sa & 3 & matrix or empty list &  & \\
\addlinespace
\hspace{3em}trend & 3 & matrix or empty list &  & \\
\hspace{3em}seasonal & 3 & matrix or empty list &  & \\
\hspace{3em}transitory & 3 & matrix or empty list &  & \\
\hspace{3em}irregular & 3 & matrix or empty list &  & \\
\hspace{2em}linearized & 2 & mts, ts, matrix &  & \\
\addlinespace
\hspace{2em}components & 2 & mts, ts, matrix &  & \\
\textbf{\hspace{1em}final} & \textbf{1} & \textbf{list} & \textbf{final}\\
\hspace{2em}series & 2 & mts, ts, matrix &  & \\
\hspace{2em}forecasts & 2 & mts, ts, matrix &  & \\
\textbf{\hspace{1em}diagnostics} & \textbf{1} & \textbf{list} & \textbf{diagnostics}\\
\addlinespace
\hspace{2em}variance\_decomposition & 2 & data.frame &  & \\
\hspace{2em}combined\_test & 2 & list & combined\_test & \\
\hspace{3em}tests\_for\_stable\_seasonality & 3 & data.frame &  & \\
\hspace{3em}combined\_seasonality\_test & 3 & character &  & \\
\hspace{2em}residuals\_test & 2 & data.frame &  & \\
\textbf{\hspace{1em}user\_defined} & \textbf{1} & \textbf{list} & \textbf{user\_defined}\\
\bottomrule
\end{longtable}\endgroup{}

\hypertarget{regarima}{%
\subsection{Regarima}\label{regarima}}

Here we can also present the output: print and graphs.

\begin{CodeChunk}

\begin{CodeInput}
R> library(RJDemetra)
R> myseries <- ipi_c_eu[, "FR"]
R> mysa <- x13_def(myseries, spec=c("RSA5c"))
R> mysa$regarima
\end{CodeInput}

\begin{CodeOutput}
y = regression model + arima (0, 1, 1, 0, 1, 1)
Log-transformation: no
Coefficients:
          Estimate Std. Error
Theta(1)   -0.5270      0.048
BTheta(1)  -0.4865      0.051

              Estimate Std. Error
Monday       -0.133839      0.164
Tuesday      -0.002384      0.163
Wednesday     0.241712      0.163
Thursday     -0.531275      0.163
Friday        0.432474      0.164
Saturday      0.152956      0.163
Leap year    -0.045977      0.501
Easter [1]   -1.094082      0.335
LS (11-2008) -8.441602      1.307
LS (1-2009)  -7.274012      1.306
LS (5-2008)  -5.020079      1.257


Residual standard error: 1.665 on 323 degrees of freedom
Log likelihood = -624.7, aic =  1277 aicc =  1279, bic(corrected for length) = 1.252
\end{CodeOutput}
\end{CodeChunk}

\hypertarget{decomposition}{%
\subsection{Decomposition}\label{decomposition}}

\hypertarget{final}{%
\subsection{Final}\label{final}}

\hypertarget{diagnostics}{%
\subsection{Diagnostics}\label{diagnostics}}

\hypertarget{user-defined}{%
\subsection{user defined}\label{user-defined}}

\hypertarget{model-specification-creation-and-modification}{%
\section{Model specification: creation and
modification}\label{model-specification-creation-and-modification}}

\hypertarget{x13}{%
\subsection{X13}\label{x13}}

\hypertarget{tramoseats}{%
\subsection{TRAMOSEATS}\label{tramoseats}}

\hypertarget{regarima-1}{%
\subsection{Regarima}\label{regarima-1}}

\hypertarget{wrong-specifications-corrections}{%
\subsection{Wrong specifications
corrections}\label{wrong-specifications-corrections}}

Parler des corrections automatiques ?

\hypertarget{manipulate-jdemetra-workspaces}{%
\section{Manipulate JDemetra+
workspaces}\label{manipulate-jdemetra-workspaces}}

Mise en garde sur ce que l'on ne peut pas faire (problèmes d'imports)



\end{document}

